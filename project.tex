\documentclass[a4paper]{article}

\usepackage{fullpage}
\usepackage{natbib}
\usepackage{graphics}
\usepackage{multirow}
\usepackage[table]{xcolor}

\title{Automatic Assessment and Stenting of Arterial Stenosis}
\author{Concours CNRS 2012 -- Projet de Recherche}
\date{R\^omulo TEIXEIRA DE ABREU PINHO}

\begin{document}

\maketitle

\begin{noindent}
{\em Obs: I am a Brazilian native who has been living in France for two years. Despite my working knowledge of the French language, I am writing this project proposal in English, for efficiency reasons.}
\end{noindent}

\begin{abstract}
Atherosclerosis figures as one of the leading cardiovascular diseases, which remain as the major cause of death in developed countries. Minimally invasive interventions are increasingly becoming the treatment option of choice, given the reduced surgical risk for the patient. A good example is the use of stents to reopen arterial stenosis or to protect the dilated walls of aneurysms from further expansion. Stent choice remains, nevertheless, as a subjective decision. An inadequate stent may migrate to another location, may exert exaggerated pressure on the arterial walls, may not properly cover the affected area, etc. 
\end{abstract}

\tableofcontents

\section{Introduction}

The choice of the stent dimensions is a direct result of the assessment of tracheal stenosis. Hence, when assessing the strictures, it is important to carefully determine their location, length, and degree of narrowing \citep{Boiselle}. The correct assessment, in conjunction with other diagnostic
tools, also aids in determining if the patient is a candidate for surgery or stent implant.

Stenosis assessment and stent choice have also been a topic of interest in the cardiovascular community. \citet{Florez} proposed a graphical model for vascular stent pose simulation. The model represents a deformable cylindrical surface that responds to an energy minimization function controlling its bending and how it is attracted to artery boundaries in image data. \citet{Bemmel} employed level-sets for the segmentation of the stenotic artery and manually chose a healthy section
of the artery as a reference to compute the level of stenosis.

Concomitantly with developments in the search for treatments for tracheal stenosis, Active Contour Models (ACMs, or snakes) and Active Shape Models (ASMs) have been important tools in computer aided diagnoses. The former is used to detect contours in an image by minimising an energy function controlling the bending and stretching of the contour and how it is attracted by image features \citep{Kass}. The latter type of models capture statistical shape variations from a set of training shapes and can be registered to clinical image data by simple adjustment of their parameters.
In addition, a distinctive characteristic of ASMs is that they only generate shapes that resemble those in the training set. This makes them more robust to noise and occlusion in comparison with other deformable models and free-form registration techniques \citep{Cootes}

Having the above concepts in mind, this thesis sets forth a decision support
system that proposes a method for assessment of tracheal stenosis and
prediction of stent length and diameter. The main idea behind this method
is to estimate the shape of the trachea of a patient as if stenosis were not
present. To this end, an ASM built solely with healthy tracheae is registered
to the CT scan of that patient. The expected result is that the registration
yields a shape that matches the healthy areas of the trachea, while at the
same time avoids the influence of the narrowed regions. Since, as mentioned
above, an ASM only generates shapes that resemble those in its training
set, the local geometric variations of the strictures, typical of stenosis, are
avoided.

Despite the motivation to use an ASM with healthy tracheae in order
to avoid the local variations of the stenosis, the strictures may have global
influence on the registration of the ASM to the image. Since variations in
calibre are usually present in the model, the strictures can eventually make
the resulting shape globally much narrower than desired. For this reason, a
novel Robust ASM Fitting procedure, called Fixed Landmarks, is proposed.
The idea behind the Fixed Landmarks is to reinforce the resistance of the
registration to the attraction incurred by the narrowed parts of the trachea.
This is achieved by keeping the parts of the estimated trachea corresponding
to regions with stenosis fixed at each iteration, relative to their location
at the previous iteration. In this way, not only are the narrowed regions
avoided, but the fixed areas incur an extra repelling force that will aid in
the estimation of the healthy tracheal wall over those regions.

When the ASM registration is finished, it is thus expected that the
resulting shape be estimation of the patient’s trachea as if stenosis were
not present. In order to assess the extent and degree of stenosis, however,
it is still necessary to segment the narrowed trachea from the image.
This is achieved using an ACM tailored for this purpose. Using the estimated
healthy trachea as a starting point, the ACM will deform it until the
boundary of the narrowed trachea is completely detected. The internal and
external forces of the ACM were therefore specifically designed to let the
model yield the best possible contour delineation with the lowest possible
shape deformation.

\section{Project Description}

The objective of the proposed project is to develop a solution for the problem of stent choice in the treatment of vascular stenoses and aneurysms. To achieve the desired result, several challenges need to be faced, among them the segmentation of the arterial tree and its surface representation, the construction of a model to predict healthy arteries from images of affected patients, an algorithm for the quantification of stenosis and aneurysms, a model for the choice of stent taking into account arteries' and stents' physical properties, and, finally, a simulation environment for the study of stent placement and blood flow. In the following, each of these aspects is discussed in more detail, and the envisioned solutions in the context of this project proposal are presented.

\subsection{Surface Representation of the Arterial Tree}

\subsection{Estimation of Healthy Arteries}

In my thesis, I proposed a method for the estimation of healthy tracheae from CT images of patients suffering from tracheal stenosis. This method is based on the construction of a statistical model \citep{Cootes} of 3D surfaces of healthy tracheae. When the model is registered to a CT scan of a patient with tracheal stenosis, it is capable of matching the parts of the tracheal wall that are healthy while at the same providing an estimation for the narrowed walls if they were healthy as well. This method showed very promising results \citep{Pinho:Trachea4}.

A natural extension of my thesis is the application of the proposed method in the cardiovascular domain. Since arteries, as well as tracheae, are objects of tubular topology, and the characteristics of their stenoses are roughly the same, it is intuitive to believe that the method will also be able to estimate the shape of healthy arteries from images of patients with arterial stenosis. This rule also tends to apply in the case of aneurysms, as long as there are enough healthy arterial regions around the location of the swelling.

I will then adapt the method I proposed for tracheal stenosis so that it can be used with narrowed and swelled arteries. I believe that in the first instance, the method developed for the trachea can be used as is. I will build a training set of healthy artery sections using segmentation methods available in the literature, for example \citep{Florez2}. Since healthy artery sections tend to appear with good contrast in the images, their segmentation is in principle a relatively straightforward task. 

The main difficulty lies in the segmentation of the arterial tree and the corresponding surface representation to be used in the statistical model. To the best of my knowledge, statistical shape models of tree-like structures have not been proposed. I intend to employ the surface representation method described above and benefit from the branch nomenclature in order to establish correspondences between the shapes of the statistical model's training set, which is an important step in the construction of the model construction. 

\subsection{Segmentation of the Arterial Tree}

The segmentation of the arterial tree has been a topic of study for quite some time. My experience with the segmentation of the trachea, however, has demonstrated that the segmentation of narrowed tube-like structures in medical images (such as arteries with stenosis) can be challenging. I will then experiment with the deformable model I proposed in my thesis, based on the snake model proposed by \cite{Kass}.

\subsection{Stent Choice}

\citep{deBeule}

\section{Integration in the Laboratory, Collaborations, Schedule}

\subsection{Laboratory}

My intention of joining the CREATIS group is on the one hand to benefit from their expertise in the several areas correlated to this project proposal. On the other hand, it will be my role to contribute to this expertise by bringing to the lab my own experience with the assessment and stenting of tracheal stenosis and applying it to the cardiovascular domain.

CREATIS have at their disposal an internal computer cluster and also the computation power of the {\bf Centre de Calcul de L'Institut National de Physique Nucl\'eaire et de Physique des Particules (CC-IN2P3)}. These will be invaluable tools for the simulation part of the proposed project. 

Most importantly, CREATIS also profits from a very multidisciplinary nature and strong links with the clinical practice, most notably that of the {\bf h\^opital neuro-cardiologique des Hospices Civiles de Lyon (HCL)} and of the {\bf Centre L\'eon B\'erard (CLB)}. This will give me the opportunity to evaluate the results of the proposed project with real patient data and to have immediate feedback from physicians. 

\subsection{Collaborations}

During my PhD in Belgium, I established a very good relationship Prof. Dr. Jan Sijbers, who was my thesis supervisor, in the VisionLab group of the University of Antwerp. The group has large experience with medical image segmentation problems and modelling of cylindrical objects, which could be interesting in the context of this project. I would like to bring the VisionLab and CREATIS together during the course of the proposed project, so that they can benefit from each other's work.

My stay in Belgium also allowed me to get acquainted with the work of the Stent Research Unit of the University of Ghent. Their work was focussed on the design of stents using numerical simulations, on which much of this project may be based. 

I also envision collaborations with the academic and industrial partners of the European Project THROMBUS, the aim of which is to evaluate the efficacy of the use of stents in neurological aneurysms. These partners, including CREATIS, combine expertise in different areas and will certainly be a good source of information for my project.

Lastly, an opportunity will be open from early 2014, with the first calls for the UE project ''Horizon 2020``. This will probably be a good chance to join partners such as those in the THROMBUS project or establish new partnerships. 

\subsection{Tentative Schedule}

\begin{table}[h]\centering
\begin{tabular}{c|c|c|c|c|c|c|c|c|}
\cline{2-9}
 & \multicolumn{2}{|c|}{Y1} & \multicolumn{2}{|c|}{Y2} & \multicolumn{2}{|c|}{Y3} & \multicolumn{2}{|c|}{Y4} \\ \cline{2-9}
 & S1 & S2 & S3 & S4 & S5 & S6 & S7 & S8  \\ \hline
\multicolumn{1}{|c|}{Apply existing model} & \cellcolor{green} & & & & & & & \\ \hline
\multicolumn{1}{|c|}{Segmentation and modelling} & & \cellcolor{green} & \cellcolor{green} & \cellcolor{green} & & & & \\ \hline
\multicolumn{1}{|c|}{Stent choice and simulations} & & & & & \cellcolor{green} & \cellcolor{green} & \cellcolor{green} & \cellcolor{green} \\ \hline
\end{tabular}
\caption{Tentative project schedule for 4 years, subdivided into semesters.}
\label{tab:schedule}
\end{table}

Table \ref{tab:schedule} shows a tentative schedule for the proposed project. The objective in the short term, roughly the first 6 months, is to try to directly apply the statistical model I proposed in my thesis to cases of arterial stenosis and aneurysms. I have reasons to believe that this step tends to be rather straightforward, requiring only few modifications to the original method, if any. The possibly biggest challenge would be the segmentation and surface modelling of the arteries, which, in the first instance, could be simplified (e.g., not taking bifurcations into account) and accomplished with existing techniques so as to yield acceptable results in a short period of time. 

The following step, to be carried out during the next 18 months, will be the extension of the statistical model with anatomical information about the arterial tree. Bifurcations will also be taken into account. 

Finally, the remaining 24 months will be dedicated to the stent choice and simulation parts. The starting point of this step will be the stent choice with a generic stent and artery model. In other words, stents' and arteries' physical properties will not be taken into account. In this way, we will be able to at least evaluate if the generic stents computed with use of the statistical model are adequate. Later, blood flow and stent design simulation results will be used to improve the statistical model and, ultimately, the automatic stent choice.  

%\section{Conclusions}

\bibliographystyle{plainnat}
\bibliography{mybib,trachea,stents,artery}

\end{document}
