\documentclass{article}

\usepackage{fullpage}
\usepackage{natbib}

\title{Automatic Assessment and Stenting of Arterial Stenosis}
\author{Concours CNRS 2012 -- Projet de Recherche}
\date{R\^omulo TEIXEIRA DE ABREU PINHO}

\begin{document}

\maketitle

\begin{flushleft}
{\em Obs: I am a Brazilian native who has been living in France for two years. Despite my working knowledge of the French language, I am writing this project proposal in English, for efficiency reasons.}
\end{flushleft}

\begin{abstract}

\end{abstract}

\tableofcontents

\section{Introduction}

The choice of the stent dimensions is a direct result of the assessment of tracheal stenosis. Hence, when assessing the strictures, it is important to carefully determine their location, length, and degree of narrowing \citep{Boiselle}. The correct assessment, in conjunction with other diagnostic
tools, also aids in determining if the patient is a candidate for surgery or stent implant.

Stenosis assessment and stent choice have also been a topic of interest in the cardiovascular community. \citet{Florez} proposed a graphical model for vascular stent pose simulation. The model represents a deformable cylindrical surface that responds to an energy minimization function controlling its bending and how it is attracted to artery boundaries in image data. \citet{Bemmel} employed level-sets for the segmentation of the stenotic artery and manually chose a healthy section
of the artery as a reference to compute the level of stenosis.

Concomitantly with developments in the search for treatments for tracheal stenosis, Active Contour Models (ACMs, or snakes) and Active Shape Models (ASMs) have been important tools in computer aided diagnoses. The former is used to detect contours in an image by minimising an energy function controlling the bending and stretching of the contour and how it is attracted by image features \citep{Kass}. The latter type of models capture statistical shape variations from a set of training shapes and can be registered to clinical image data by simple adjustment of their parameters.
In addition, a distinctive characteristic of ASMs is that they only generate shapes that resemble those in the training set. This makes them more robust to noise and occlusion in comparison with other deformable models and free-form registration techniques \citep{Cootes}

Having the above concepts in mind, this thesis sets forth a decision support
system that proposes a method for assessment of tracheal stenosis and
prediction of stent length and diameter. The main idea behind this method
is to estimate the shape of the trachea of a patient as if stenosis were not
present. To this end, an ASM built solely with healthy tracheas is registered
to the CT scan of that patient. The expected result is that the registration
yields a shape that matches the healthy areas of the trachea, while at the
same time avoids the influence of the narrowed regions. Since, as mentioned
above, an ASM only generates shapes that resemble those in its training
set, the local geometric variations of the strictures, typical of stenosis, are
avoided.

Despite the motivation to use an ASM with healthy tracheas in order
to avoid the local variations of the stenosis, the strictures may have global
influence on the registration of the ASM to the image. Since variations in
calibre are usually present in the model, the strictures can eventually make
the resulting shape globally much narrower than desired. For this reason, a
novel Robust ASM Fitting procedure, called Fixed Landmarks, is proposed.
The idea behind the Fixed Landmarks is to reinforce the resistance of the
registration to the attraction incurred by the narrowed parts of the trachea.
This is achieved by keeping the parts of the estimated trachea corresponding
to regions with stenosis fixed at each iteration, relative to their location
at the previous iteration. In this way, not only are the narrowed regions
avoided, but the fixed areas incur an extra repelling force that will aid in
the estimation of the healthy tracheal wall over those regions.

When the ASM registration is finished, it is thus expected that the
resulting shape be estimation of the patient’s trachea as if stenosis were
not present. In order to assess the extent and degree of stenosis, however,
it is still necessary to segment the narrowed trachea from the image.
This is achieved using an ACM tailored for this purpose. Using the estimated
healthy trachea as a starting point, the ACM will deform it until the
boundary of the narrowed trachea is completely detected. The internal and
external forces of the ACM were therefore specifically designed to let the
model yield the best possible contour delineation with the lowest possible
shape deformation.
\section{Estimation of Healthy Arteries}

When physicians
assess the stenosis, it is intuitive to imagine that they try to guess
from the available data how the patient’s trachea would appear if the narrowing
were not present. This image of the healthy trachea is the basis to
determine the location, length, and degree of the stricture. The image of the calibre and curvature of the estimated healthy shape also aid in choosing
the type and dimensions of the stent to be used, or even the shape the
trachea should have after a reconstruction surgery.

Inspired by this procedure, the proposed method aims at providing to
the physician the desired healthy shape. This shape not only represents
the necessary visual aid in the diagnostic and pre-operative processes, but
also enables the automatic assessment of the stenosis and suggestion of
patient-specific stent dimensions. The estimation of the healthy shape of
the trachea is obtained thanks to the use of an Active Shape Model of
healthy tracheas and to a new algorithm to register such model to CT
data.

An Active Shape Model (ASM), as defined by \citet{Cootes}, is
a statistical model of shapes whose objective is to capture the geometric
variation present in a set of training shapes. This model is then capable
of generating new shape instances which are constrained by the statistical
variation of the training set, referred to as the allowable shape domain.
As a result, a defining characteristic of ASMs, which is an advantage with
respect to other deformable models, is that they can only generate shapes
that resemble those in the training set.

This characteristic is the motivation to use an ASM of healthy tracheas
to estimate the healthy tracheal shape of a patient with stenosis. Since the
model contains only instances of healthy tracheas, local geometric variations
typical of stenosis cannot be generated. The details of the construction and
use of the proposed ASM are presented in the following.




\bibliographystyle{plainnat}
\bibliography{trachea}

\end{document}
