\documentclass{article}

\usepackage[subsectionbib]{bibunits}
\usepackage{fullpage}
\usepackage{multirow}
%\usepackage[sort]{natbib}
 \usepackage{longtable}
 
\title{Curriculum Vitae}
\author{Concours CNRS 2012 -- Section 7}
\date{R\^omulo (TEIXEIRA DE ABREU) PINHO}

\begin{document}

\maketitle


\section{Contact Information}

40, Rue Colin\\
Villeurbanne, 69100, France.\\
Mobile Number: +33-(0)-638305900\\
e-mail: romulo.pinho@lyon.unicancer.fr\\
URL: www.creatis.insa-lyon.fr/rio/RomuloPinho

\section{Biographical Data}

Birth date: March, 3rd, 1977\\
Place of Birth: Petr\'opolis, RJ, Brazil\\
Citizenship: Brazilian\\
Marital status: Married

\section{Current Position}

\begin{longtable}{p{0.25\textwidth} p{0.6\textwidth}}
January, 2011 \newline present & Post-doc Research Engineer \newline
Centre L\'eon B\'erard, Lyon, France \newline
Design, implementation, and deployment of a medical imaging system for image guided
radiation therapy against lung cancer. The aim is to carry out a clinical trial for the use of new
image registration techniques and their influence on radiation therapy against moving tumours. 
Multi-platform development using C++, CMake, ITK, VTK, QT, Python, and shell scripts.\\
\\
September, 2011 \newline present & Teacher assistant \newline
Subject: Operating Systems \newline
(in French, 16h, practice sessions) \newline
Institution: INSA-Lyon\\
\\
September, 2011 \newline present & Teacher assistant \newline
Subject: Object Oriented Programming \newline
(in English, 45h, theory and practice sessions) \newline
Institution: INSA-Lyon\\
\\
\end{longtable}


\section{Education}

\begin{longtable}{l p{0.6\columnwidth}}
November, 2010 & Ph.D. in Physics. \newline 
Universiteit Antwerpen, Antwerp, Belgium \newline 
Specialization: Medical Image Processing \newline 
Thesis: A Decision Support System for the Assessment and Stenting of Tracheal Stenosis. \\
\\
May, 2001 & M.Sc. in Software Engineering. \newline 
Universidade Federal do Rio de Janeiro, Rio de Janeiro, Brazil \newline 
Specialization: Computer Graphics \newline 
Thesis: Three Dimensional Reconstruction of Head Models from Magnetic Resonance Images. \\
\\
September, 1998 & B.Sc.  in Computer Science \newline 
Universidade Federal Fluminense, Niter\'oi, Brazil \newline 
Specialization: Computer Science.\\
\end{longtable}


\section{Honours, Awards and Certificates}

\begin{tabular}{p{0.25\columnwidth} p{0.6\columnwidth}}
Las Vegas, April, 2005 & ScrumMaster Certificate.
\end{tabular}

\section{Languages}

\begin{tabular}{l p{0.6\columnwidth}}
Portuguese & native\\
English & fluent (non-native)\\
French & intermediate level\\
Dutch & basic level\\
\end{tabular}

%\renewcommand{\refname}{Contributions}
%\begin{bibunit}[unsrt]
%\bf{Book Chapters} \\
%\nocite{Pinho:Cache1}
%\nocite{Pinho:Cache2}
%\bf{Journals} \\
%\nocite{Pinho:Airways1}
%\nocite{Pinho:Airways2}
%\bf{Conference Papers (full paper)} \\
%\nocite{Pinho:Trachea1}
%\nocite{Pinho:Trachea2}
%\nocite{Pinho:Trachea3}
%\nocite{Pinho:Trachea4}
%\nocite{Pinho:Trachea5}
%\nocite{Pinho:Trachea6}
%\nocite{Pinho:Trachea7}
%\putbib[mybib]
%\end{bibunit}

%\bibliographyunit[\section]
\section{Scientific Publications}

\renewcommand{\refname}{Journal papers}
\begin{bibunit}[unsrt]
\nocite{Pinho:Trachea7}
\nocite{Pinho:Trachea4}
%\nocite{Pinho:Cache2}
\putbib[mybib]
\end{bibunit}

\renewcommand{\refname}{Book chapters}
\begin{bibunit}[unsrt]
\nocite{Pinho:Trachea5}
\putbib[mybib]
\end{bibunit}

\renewcommand{\refname}{Conference proceedings (full paper)}
\begin{bibunit}[unsrt]
\nocite{DELM-11}
\nocite{PINH-11}
\nocite{RIT-11b}
\nocite{Pinho:Trachea6}
\nocite{Pinho:Airways2}
\nocite{Pinho:Trachea3}
\nocite{Pinho:Trachea2}
\nocite{Pinho:Cache1}
\nocite{Pinho:Trachea1}
\nocite{Pinho:Airways1}
\putbib[mybib]
\end{bibunit}

\renewcommand{\refname}{Conference proceedings (abstract)}
\begin{bibunit}[unsrt]
\nocite{Bernat:Clavicle}
\nocite{Huysmans:Clavicle}
\nocite{Pinho:Trachea8}
\putbib[mybib]
\end{bibunit}


\section{Academic Experience}

% \subsection{Teaching}
% \begin{tabular}{p{0.25\columnwidth} p{0.6\columnwidth}}
% September, 2011 \newline present & Teacher assistant \newline
% Subject: Operating Systems \newline
% (in French, 16h, practice sessions) \newline
% Institution: INSA-Lyon\\
% \\
% September, 2011 \newline present & Teacher assistant \newline
% Subject: Object Oriented Programming \newline
% (in English, 45h, theory and practice sessions) \newline
% Institution: INSA-Lyon\\
% \end{tabular}

\subsection{Supervision of Theses}
\begin{tabular}{p{0.25\columnwidth} p{0.6\columnwidth}}
January, 2009 \newline June, 2009 & B.Sc. Thesis of Stijn Peeters (co-supervision) \newline
Title: Automatic segmentation of the airways: Morphological study of the trachea (in -Dutch) \newline
Institution: University of Antwerp\\
\\
January, 2009 \newline June, 2009 & B.Sc. Thesis of Sten Luyckx (co-supervision) \newline
Title: Automatic segmentation of the airways in 3D CT using cylindrical shape modelling (in Dutch) \newline
Institution: University of Antwerp\\
\end{tabular}

\subsection{Research Projects}
\begin{tabular}{p{0.25\columnwidth} p{0.65\columnwidth}}
April, 2010 \newline December, 2010 & CIMI -- IBBT \newline 
Responsible for and lead researcher of Task 2.1 of WP 3 \newline
Development of cache and pre-fetching techniques for out-of-core processing and visualization of large microscopic images. \\
\\
April, 2010 \newline December, 2010 & Segmentor -- IBBT \newline
Researcher \newline
Development of semi-automatic segmentation techniques based on haptics and deformable models for the segmentation of tracheal stenosis.
\end{tabular}


\section{Professional Experience}

\begin{longtable}{p{0.25\columnwidth} p{0.65\columnwidth}}

January, 2001 \newline April, 2006 & TV Systems Software Engineer/Researcher \newline
R\&D Department, TV Globo Ltd. \newline
Design and development of software for general acquisition,
compression, storage, processing, and transmission of
video and audio signals. Applications were implemented in C/C++ for Win32 and
Linux platforms, using various SDKs and libraries
including DirectShow SDK, Windows Media SDK, Matrox
DigiSuite SDK, MFC, Video4Linux, MySQL.\\
\\
October, 2001 \newline May, 2003 & Software Engineer/Researcher\newline
LAMEC -- Laboratory of Computational Mechanics,\newline
Federal University of Rio de Janeiro \newline
Design and development of a C++, OpenGL/QT-based scientific
visualization and geometric modelling system to build 3d 
models out of simple 3d primitives, to be used in numerical computations.\\
\\
August, 2000 \newline December, 2000 & Software Engineer\newline
Publintel S. A.\newline
Development and maintenance of GIS applications. Translation
of existing GUI code to the Win32 environment. Design and
implementation of a logistics database for automatic vehicle 
localization, using SQL-Server, C++, STL, MFC, and ADO.\\
\\
March, 2000 \newline August, 2000 & Technology Researcher/Programmer\newline
Montreal Inform\'atica\newline
Design and implementation of Speech Recognition and
biometrics (fingerprint identification) systems. These 
applications were implemented in C++ and Delphi, using 
the ORACLE DBMS.\\
\\
January, 2000 \newline March, 2000 & System Analyst/Programmer\newline
Delta de Friburgo\newline
I took part in the design of a profit optimization
system for the Brazilian Petroleum Company
(PETROBRAS), using C++ and the ORACLE DBMS.\\
\\
April, 1999 \newline January, 2000 & Programmer\newline
Brainstorming Ass. e Plan. em Inform\'atica\newline
Lead programmer of a database system to control 
the use of the necessary material for the repair of ships, weapons 
and for general services inside a Brazilian Navy base in Rio de Janeiro.
Development in Delphi/ORACLE.\\
\\
January, 1998 \newline January, 1999 & Game Programmer\newline
Z-Movie Studio.\newline
Development of 3D educational computer games. Design and
implementation of 3D engine, character animation, GUIs and game logic.
Implementations in C++ for DirectX 6.0.\\
\\
October, 1996 \newline January, 1998 & Programmer\newline
ADD-Labs -- Universidade Federal Fluminense (Laboratory of Active Document Design)\newline
Development of a prototype Virtual Reality system built in C/C++ to the Brazilian Petroleum 
Company (PETROBRAS). The objective was to provide the user with a 3D 
visualization of oil extraction fields.\\
\end{longtable}

%\section{Other Relevant Professional Experience}

%\begin{tabular}{p{0.25\columnwidth} p{0.65\columnwidth}}
%	& 10-year experience working with C/C++;
%	& 4-year experience working with OpenGL;
%	& 3-year experience working with DirectShow
%	(filter user experience and basic filter development
%	experience � DirectX 7.0, 8.0, 8.1);
%	& 2-year experience working with Delphi;
%	& 3-year experience working with ORACLE, MySQL
%	and PL/SQL;
%	& 1-year experience working with DirectDraw and
%	Direct3D (DirectX 6.0);
%	& Good knowledge of MFC, STL and general Win32
%	programming;
%	& Good knowledge of Linux programming;
%\end{tabular}


\end{document}